\newpage
\section{Conclusion}
\label{sec:conclusion}
In this review, we have performed a systematic literature review following a SLR approach to review \samplesize papers on the topic of probabilistic AI in finance. We examined these papers across dimensions such as model type, output, asset class, and uncertainty type. Additionally, we provided insights into the geographical distribution of research, contributor backgrounds, and the historical development of the field. Our findings suggest that most articles on probabilistic AI claim to enhance point predictions, and few articles have an explicit focus on improving uncertainty estimation within finance. Moreover, probabilistic AI offer valuable capabilities for financial modeling, including non-parametric distribution estimation, separation of uncertainty types, and capturing non-linear dynamics. However, the lack of comprehensive benchmarking, particularly against traditional models, limits conclusions about their true value. 

An important implication of our findings is the need for more interdisciplinary collaboration. Analysis of author backgrounds indicates that research in this area is largely dominated by computer scientists, with relatively limited participation from financial experts. As a result, computer scientists often lack the domain-specific knowledge needed to effectively model financial problems, while financial researchers—despite being better positioned to address such challenges—have seldom adopted probabilistic AI techniques, likely due to technical barriers. This review serves as a starting point for bridging these divides, guiding financial researchers in adopting these methods and helping computer scientists better frame their approaches within the financial context.

Finally, our review highlights the immaturity of the field. Most of the reviewed articles have been published very recently, with few building on each other, and relatively few achieving high standards in modeling, testing, and interpretation. Most authors do not disclose code, which hinders reproducibility and the possibility for building on previous work. While the field is promising, it would benefit greatly from following the suggested approaches for standardizing testing, benchmarking, and interpreting estimates to fully leverage the potential of probabilistic AI for uncertainty quantification in financial applications.