\section{Conclusion}
\label{sec:conclusion}
In this review, we have performed a systematic literature review following a SLR approach to review \samplesize papers on the topic of probabilistic AI in finance. We examined these papers across dimensions such as model type, output, asset class, and uncertainty type. Additionally, we provided insights into the geographical distribution of research, contributor backgrounds, and the historical development of the field. Our findings suggest that most articles on probabilistic AI claim to enhance point predictions. Additionally, probabilistic AI offer valuable capabilities for financial modeling, including non-parametric distribution estimation, separation of uncertainty types, and capturing non-linear dynamics. However, the lack of comprehensive benchmarking, particularly against traditional models, limits conclusions about their true value.

An important implication of our findings is the need for more interdisciplinary research. The descriptive statistics of author backgrounds reveal that the field is predominantly driven by computer science researchers, with limited involvement from financial researchers. This creates a gap where computer scientists often lack the domain knowledge necessary to properly model financial problems, while financial researchers, who are better equipped to address domain-specific challenges, have largely not adopted probabilistic AI techniques, perhaps due to technical barriers. This review provides a starting point for bridging this divide, offering guidance to financial researchers on adopting these methods and helping computer scientists better frame their approaches within the financial context.

Finally, our review highlights the immaturity of the field. Most of the reviewed articles have been published very recently, with few building on each other, and relatively few achieving high standards in modeling, testing, and interpretation. Most authors do not publish the code either, which hinders reproducibility and the possibility for building on each other. While the field is promising, it would benefit greatly from following the suggested approaches for standardizing testing, benchmarking, and interpreting estimates to fully leverage its potential in financial applications.