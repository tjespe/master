\label{sec:abstract}
%Recent advances in AI and Finance show that even when the technical methods become more sophisticated stock prices remain difficult to predict. ..

%in this systematic literature review, we ...

%systematic literature review

%"probabilistic AI to improve uncertainty estimates in financial time series" , " uncertainty quantification", "volatility", "Artificial intelligence" and "machine learning", "aleatoric and epistemic uncertainty", "Financial time series" , "64 papers", "traditional models", "financial risk measures", "To ensure generalizability and validity we will use data from a broad range of markets.", ARIMA, GARCH, 

\normalsize

In this systematic literature review, we explore the application of probabilistic AI models to enhance uncertainty estimation in financial time series forecasting. Unlike traditional models, probabilistic AI approaches not only make predictions but also quantify the uncertainty of those predictions or provide full distributional forecasts. This capability allows for improved assessment of both epistemic and aleatoric uncertainty. We systematically review recent advancements across various probabilistic frameworks, including Bayesian neural networks, Gaussian processes, and generative models, applied to diverse financial assets. Findings are categorized by model type, output format, asset class, and type of uncertainty quantified.

Our analysis reveals significant gaps in the field, including a lack of standardized benchmarks, evaluation metrics, and interdisciplinary collaboration, as well as limited financial interpretation of results. Few studies rigorously assess the validity of uncertainty estimates, and even fewer benchmark against traditional models such as GARCH, hindering definitive conclusions about the performance of probabilistic models.

We conclude that while probabilistic AI has significant potential to enhance risk assessment and financial decision-making through better uncertainty quantification, this potential remains largely underutilized. To address these gaps, we propose a standardized evaluation framework and advocate for greater interdisciplinary collaboration to advance the application of probabilistic AI in financial forecasting.