\label{sec:abstract}
%Recent advances in AI and Finance show that even when the technical methods become more sophisticated stock prices remain difficult to predict. ..

%in this systematic literature review, we ...

%systematic literature review

%"probabilistic AI to improve uncertainty estimates in financial time series" , " uncertainty quantification", "volatility", "Artificial intelligence" and "machine learning", "aleatoric and epistemic uncertainty", "Financial time series" , "64 papers", "traditional models", "financial risk measures", "To ensure generalizability and validity we will use data from a broad range of markets.", ARIMA, GARCH, 

\normalsize In this systematic literature review, we examine the application of probabilistic AI models for improved uncertainty estimates in financial time series forecasting. Probabilistic AI models are capable of not only making predictions, but also quantifying the uncertainty of the predictions or provide full distributional forecasts, enabling improved estimation of both epistemic and aleatoric uncertainty. We systematically analyze recent advancement in the field across several probabilistic frameworks, including Bayesian neural networks, Gaussian processes, and generative models, applied to a range of financial assets. The review categorize findings by type of model, model output, predicted asset class and type of uncertainty quantified. In general, we find a lack of standardized benchmarks, evaluation metrics, limited interdisciplinary collaboration and a lack of financial interpretation of the results. Few papers assess the validity of the uncertainty estimates, and even fewer benchmark against traditional models like GARCH, limiting conclusions on true model performance. Our findings emphasize that probabilistic AI's potential to improve risk assessment and financial decision-making by quantifying uncertainty remains underutilized. However, we propose a standardized evaluation framework for more rigorous testing and recommend interdisciplinary collaboration to advance the maturity of probabilistic AI in financial applications.