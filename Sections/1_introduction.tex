\section{Introduction}
\label{sec:introduction}
%Points to include: Motivation, Overview of research area, context, research question, contribution, structure of paper.... : 
% Motivation
%\textit{Briefly touch upon the need for better uncertainty estimates in finance and how probabilistic AI models have started to fill gaps left by traditional econometric approaches.}
In recent years, there has been a significant increase in the application of artificial intelligence (AI)\footnote{See Table \ref{table:abbreviations} for a list of all abbreviations used in this paper.
} and machine learning (ML) models within finance, motivated by AI models' potential to provide price forecasts despite an efficient market \textcite{sezer2020financial}. However, a major drawback of AI models is that they are to a large extent ``black boxes'' making it difficult to understand how to interpret predictions and buy/sell recommendations. Probabilistic AI models partly alleviates this issue, as they can provide well-calibrated probabilities for different scenarios, allowing for more informed decision-making and sophisticated investment strategies. However, as this review will show, the usefulness of such estimates vary widely based on implementation.

Traditional econometric models such as autoregressive integrated moving average (ARIMA) by \textcite{boxJenkins2016time}, originally published in 1970, and autoregressive conditional heteroskedasticity (ARCH) by \textcite{Engle1982ARCH} with its extension to generalized autoregressive conditional heteroskedasticity (GARCH) by \textcite{BOLLERSLEV1986GARCH} has been fundamental in financial forecasting and used to capture volatility in financial time series. GARCH models have proven highly effective in financial applications; however, its parametric and linear structure restricts its capacity to capture the non-linear and complex interactions often found in financial time series \parencite{sezer2020financial}. This limits accuracy in capturing risk as some signals are ignored. AI models have demonstrated the ability to capture complex patterns in financial data, and applying these capabilities to enhance uncertainty estimates could be valuable for improving decision-making, optimizing portfolios, and identifying arbitrage opportunities in derivative pricing, among other applications.

Supervised machine learning models like Long Short-Term Memory networks (LSTM) and Convolutional Neural Networks (CNN) have recently emerged as an alternative to handle non-linearity within data for financial prediction \parencite{Tang2022Survey}. However, these models have mainly been applied to provide point predictions for returns, rather than full conditional probability distributions, limiting the interpretability and usefulness of the predictions.

% Context
%\textit{Define probabilistic AI and its relevance to uncertainty quantification for financial time series. Mention prominent models and methods.}
\textbf{Context}\\
Probabilistic AI often refers to a Bayesian approach to machine learning, where probabilistic theory is used to infer plausible models to explain observed data \parencite{Ghahramani2015}. However, to capture all the research relevant for improving uncertainty estimates in finance, we extend the definition to include any ML/AI models that can quantify the uncertainty in their predictions or produce full distributional forecasts. These models have the potential to provide better understanding of potential future states and risk.

Probabilistic AI is a rapidly evolving area, as this review will demonstrate, and includes models like Bayesian neural networks, transformers, Variational Auto-Encoders, Gaussian processes and Hidden Markov models, among others. The number of articles within the field has risen sharply the last years, with many promising results in many cases. However, the majority of AI applications in finance have focused on generating precise point forecasts for prices or returns, and the potential usefulness of probabilistic predictions has received less attention \parencite{sezer2020financial}.

When discussing quantification of uncertainty, it is important to distinguish between epistemic (model-driven) and aleatoric (underlying) uncertainty. Epistemic uncertainty arises from the model's limitations in capturing data patterns, reducible through better models or more data. Aleatoric uncertainty refers to inherent data randomness, equivalent to underlying volatility in finance. It is influenced by unpredictable market behavior, economic events, and investor sentiment, beyond the reach of any model. \parencite[7,34]{pml1Book, KIUREGHIAN2009105, hullermeier2021aleatoric} A sophisticated probabilistic AI model can quantify both types of uncertainty, as well as model how the aleatoric uncertainty (volatility) changes over time. However, as this review will show, only a minority of proposed models fully exploit this possibility.


% Overview of research area

\textbf{Overview of Research Area}\\
There have been several studies and reviews over the last years on the application of AI and machine learning for financial time series prediction. \textcite{preethi2012stock}, \textcite{soni2015cloud} and \textcite{gandhmalstockmarket2019} all review machine learning techniques for stock market trend or point predictions. The studies review literature up to 2012, 2015, and 2018 respectively. \textcite{soni2015cloud} demonstrate that neural networks outperform traditional methods in capturing non-linear patterns in financial data, providing accurate forecasts. Similarly, \textcite{gandhmalstockmarket2019} conclude that ANN models and fuzzy-based techniques are the most promising among the reviewed machine learning approaches for accurate stock market predictions. \textcite{shi2019soft} review soft computing approaches for stock market forecasting, also finding ANN architectures to consistently outperform other machine learning models in point prediction accuracy. 

A newer study was conducted by \textcite{Khattak2023SurveyAIModels}, providing an in-depth review of machine learning methods used to forecast various financial assets between 2018 and 2023. The study finds new hybrid integrations of LTSM and SVM architectures to be the most effective in financial market predictions, outperforming traditional models in point accuracy.

While there is a growing body of literature on AI and machine learning for financial applications, the aforementioned reviews focus solely on point prediction accuracy. Even though both epistemic and aleatoric uncertainty are essential components of financial decision making, fewer reviews have been conducted on the topic of uncertainty quantification in a financial context using machine learning methods or probabilistic models specifically. \textcite{abdar2021ReviewUQ} conduct a thorough review on uncertainty quantification in deep learning techniques, discussing the advantages and disadvantages of several models, but do not focus on financial time series predictions. The authors conclude that Deep Ensembles and Bayesian Neural Networks show promising capabilities for uncertainty quantification distinguishing between aleatoric and epistemic uncertainty, possibly applicable to financial forecasting, but find that lack of standardized benchmarks make it difficult to compare frameworks.

\textcite{Blasco_et_al_2024} conduct a survey on uncertainty quantification using deep learning techniques in financial time series. The study shows that most articles do not distinguish between aleatoric and epistemic uncertainty, and few authors perform analysis on the financial implications of predictive uncertainty. However, they limit their focus to deep learning techniques, do not assess the entire probabilistic AI field, and the focus on how to use and interpret the predictions in a financial context is limited. The survey includes literature on deep learning up to 2022, but as we show in this review, production has seen an explosive increase over the last two years. 

In light of the rapid increase in published articles on probabilistic AI over recent years and the scarcity of reviews focused on uncertainty quantification for various financial time series, this review aims to address these gaps by synthesizing the current state-of-the-art research. Additionally, prior financial studies have predominantly concentrated on stock markets; in response, this review includes a range of asset classes, examining how uncertainty quantification varies across categories. Given that the defining feature of probabilistic AI models is their capacity to provide predictions with associated uncertainties, this review places substantial emphasis on interpreting these uncertainty estimates within a financial context. Finally, existing research reveals a lack of consensus regarding the evaluation of uncertainty estimates produced by probabilistic models. Accordingly, we summarize current evaluation practices and, based on this, propose a clear framework for future work. 





% Reserach questions
\textbf{Research Questions}\\
To gain an understanding of the existing research, address the gaps identified in the literature, and advance the understanding of the use of probabilistic AI for uncertainty estimates in financial time series, this review seeks to answer the following key research questions: 
\begin{enumerate}
    \item Summarize to what extent and in what way existing research are using uncertainty estimates from probabilistic AI models as measures of volatility or model uncertainty, or as financial risk measures for financial time series.
    \item Analyze researchers' motivation for making predictions with uncertainty and how it differs for different asset classes.
    \item Compare how promising probabilistic models' capabilities are compared to other machine learning models and traditional econometric models in uncertainty estimation.
    \item Investigate probabilistic models' ability to provide improved understanding of risk and volatility compared to econometric models.
    \item Identify the metrics used for assessing probabilistic AI models and assess what the most appropriate metrics for assessing the quality of the produced uncertainty estimates is.
    \item Analyze how probabilistic AI models can be used to construct financial risk measures such as Value at Risk (VaR) and Expected Shortfall (ES).
    \item Identity possible areas for further research.
\end{enumerate}

We will answer the questions directly in Section \ref{sec:discussion}, but they will shape the research and literature search and the presented results in Section \ref{sec:result_and_discussion}. 

 
% Contributions
% \textit{How review addresses the specific applications in finance and recent advancements in probabilistic AI}
\textbf{Contributions}\\
The literature on probabilistic AI applications in finance is limited, and this review aims to fill the gap by focusing on recent advances in probabilistic AI and application within fiance due to the novelty and rapid evolution of probabilistic AI. Specifically, the review explores how probabilistic AI model can improve uncertainty estimates, and clarify the potential strengths and limitations highlighted in literature of such models in a financial context. Given the novelty and the rapid evolution, many researchers might have limited knowledge of the field and its potential. This review shines light on whether adopting and further researching machine learning for uncertainty estimates in finance is useful or not. Furthermore, we will discuss whether it is appropriate to interpret quantified uncertainty from AI models as volatility estimates and how to benchmark volatility estimates from probabilistic AI models. We identify which probabilistic machine learning models are used in the field and are most promising for further research and highlights areas for future research and where advancements in uncertainty quantification is needed. 

Ultimately, the review will serve as an overview of the field and guide for researchers who want to apply or advance probabilistic AI in finance. 


% Concluding paraghaph outlining the structure of the paper 
\textbf{Structure of the Paper}\\
The literature review is structured as follows: Section \ref{sec:methodology} covers the Methodology, detailing the review and analysis process. Section \ref{sec:result_and_discussion} presents the Results and Discussion, including descriptive statistics and evaluation across different dimensions. Finally, Section \ref{sec:conclusion} provides the Conclusion, summarizing key findings and suggestions for future research. 