\section{Introduction}
\label{sec:1_introduction}



% ==================== Motivation and Problem Statement ==================== %
\subsection{Motivation and Problem Statement}
\label{sec:motivation_problem_statement}
Financial markets and the tradable assets within them are subject to inherent uncertainty, posing persistent challenges for investors, policymakers and financial institutions. Accurately quantifying and managing uncertainty is thus crucial for informed financial decision-making and regulatory compliance. Particularly, tail risk measures such as Value at Risk (VaR) and Expected Shortfall (ES) have become essential for assessing risk associated with financial assets, quantifying potential losses in worst-case market scenarios, and guiding capital allocation. Recently, regulatory bodies and the broader financial literature have shifted their focus towards ES rather than VaR, largely influenced by the Basel III guidelines \parencite{basel2011, basel2019}. This shift is motivated by ES' superior ability to capture extreme tail events, experienced particularly during the global financial crisis.

Historically, financial forecasting and tail risk estimation have predominantly relied on econometric models, such as the Generalized Autoregressive Conditional Heteroskedasticity (GARCH) by \textcite{bollerslev1986garch} and its variants. Although proven robust, traditional econometric models rely on linear dependency structures and impose restrictive parametric assumptions about distributional forms—such as Gaussian or Student-t—that often fail to adequately capture the true dynamics of financial returns and their conditional variability over time, including asymmetry and heavy tailedness \parencite{Cont2001}. Advancements in computational power and increased data availability has fostered increased adoption and research interest in artificial intelligence (AI) and machine learning (ML) approaches, such as Gradient Boosting Machines (GBMs) and deep learning architectures like Long short-term memory networks (LSTMs) and Transformers, due to their flexibility and ability to capture non-linear dependencies \parencite{Sezer2020}. However, in standard implementations these ML approaches typically provide and have been used mainly for deterministic point forecasts, with limited insight into predictive uncertainty. This, along with limited interpretability make it difficult to assess the reliability and the practical usefulness of their predictions in risk-sensitive financial settings \parencite{Engelberg2009}. 

Addressing these limitations, there has been a recent surge of interest in probabilistic AI models capable of explicitly quantifying predictive uncertainty or produce full distributional forecasts rather then mere point predictions \parencite{eggen2025probabilistic}. Several of these models, including those employed in this study, can under an appropriate modeling framework produce non-parametric conditional distributions, allowing the data to dictate the shape of the distributions. This flexibility can help capturing heavy tails, skewness and conditional volatility dynamics better, and allows for potentially arbitrary distributional forms conditioned on input parameters. Furthermore, with suitable modeling choices, probabilistic AI methods can distinguish between epistemic uncertainty—arising from model limitations—and aleatoric uncertainty—caused by inherent randomness in the data. This enables researchers and financial risk managers to better understand the nature of uncertainty in predictions, facilitating more informed and risk-aware decision-making. By embedding all this information within predictions, probabilistic AI has the potential to enhance various financial tasks, including trend forecasting, portfolio optimization, option pricing, and—investigated in this thesis—tail risk estimation. Despite this theoretical potential, the practical applicability and true efficacy of probabilistic AI for financial tail risk estimation remains underexplored in the literature, and the few existing studies often lack rigorous benchmarking and testing procedures, complicating the derivation of meaningful conclusions \parencite{eggen2025probabilistic}. 

This thesis aims to address these gaps by systematically evaluating the performance of selected probabilistic AI models in estimating one day-ahead ES for individual stock returns within the Dow Jones Industrial Average (DJIA), using implied volatility (IV) and realized variance (RV) measures as explanatory variables. Specifically, we implement two deep learning approaches in a probabilistic framework by utilizing Mixture Density Networks (MDN); LSTM-MDN and Transformer-MDN. These models forecast full, non-parametric conditional return distributions, from which ES is calculated analytically. We evaluate the quality of both the full predicted distributions and the tail risk measures derived from them. Furthermore, we benchmark these models against traditional econometric approaches and other machine learning models. Ultimately, this thesis aims to determine whether probabilistic AI can enhance uncertainty quantification and improve tail risk estimation in financial time series, by addressing key limitations in existing approaches. To the best of our knowledge, this is the first study using a full non-parametric distributional forecast from probabilistic models in a MDN framework to estimate ES for financial time series. 



% ==================== Research Questions ==================== %
\subsection{Research Questions}
\label{sec:research_questions}

To address the motivations outlined above this thesis seeks to investigate the following research questions: 
\begin{enumerate}[label=RQ\arabic*:, leftmargin=1.4cm]
    \item To what extent do probabilistic AI models improve uncertainty quantification in financial time series forecasting and how effectively do they forecast full conditional return distributions for individual stocks?
    \item How adequately and accurately can tail risk measures like Expected Shortfall (ES) and Value-at-risk (VaR) be estimated from probabilistic AI models compared to traditional econometric and machine learning methods?
    
   % \item To what extent do probabilistic AI models improve uncertainty quantification in financial risk estimation, specifically in capturing aleatoric and epistemic uncertainty?
    \item What are the practical implications and limitations of using probabilistic AI models for uncertainty quantification and tail risk estimation in financial risk management? 
    \item How do IV and RV compare in their ability to predict one-day-ahead ES for individual equities? 
    \item To what extent, if any, does employing a Transformer architecture over LSTM improve predictive performance and ES estimation, and how do these architectures differ in their ability to capture complex volatility dynamics in financial returns?
\end{enumerate}

These questions will be addressed throughout the thesis and shape the presented research, before they will be answered directly in Section \ref{sec:7_discussion}.   


% ==================== Contributions ==================== %
\subsection{Contributions}
\label{sec:contributions}

This research makes several key contributions to the field of financial forecasting and risk management by leveraging probabilistic AI models for full distributional forecasts and tail risk estimation. The primary contributions are as follows:

\begin{enumerate}
    \item \textbf{Clear Demonstration of Probabilistic AI’s Superiority}: This study is among the first to conclusively demonstrate the superiority of probabilistic AI models over traditional econometric and machine learning approaches in forecasting conditional distributions and tail risk in equity markets. Prior work has either lacked rigorous testing and benchmarking or failed to establish consistent outperformance \parencite{eggen2025probabilistic}.
    \item \textbf{Flexible Distributional Modeling}: The proposed models—LSTM-MDN and Transformer-MDN—generate conditional distributions without relying on restrictive parametric assumptions. This flexibility improves forecast accuracy by capturing the well-documented heavy-tailed nature of financial returns. Only a handful of studies have proposed probabilistic AI models with this ability before, and all but one of them lack rigorous testing \parencite{eggen2025probabilistic}.
    \item \textbf{Uncertainty Quantification and Decomposition}: We systematically quantify and disentangle aleatoric (data-driven) and epistemic (model-driven) uncertainties, offering deeper insights into the sources of uncertainty and enhancing the interpretability and actionability of forecasts.
    \item \textbf{Robust Empirical Benchmarking}: A comprehensive benchmarking framework is introduced, comparing probabilistic AI models with conventional econometric and state-of-the-art machine learning methods. Building on the booster models of \textcite{moen2024forecasting} (XGBoost, LightGBM, CatBoost), we adapt them for a new asset class and a multi-asset setting. Evaluation includes rigorous distributional accuracy metrics, statistical adequacy tests, backtesting procedures, and tail risk scoring, establishing a new standard for model evaluation.
    \item \textbf{Architectural Comparison of AI Models}: We perform a systematic comparison of LSTM-MDN and Transformer-MDN architectures to assess their respective strengths in probabilistic forecasting. This directly contributes to ongoing debates regarding the effectiveness of Transformer-based models in financial time series analysis, especially in terms of predictive accuracy and tail risk estimation.
    \item \textbf{Model Explainability and Practical Transparency}: Acknowledging the critical need for transparency in financial risk management—particularly given critiques of machine learning as a 'black box'—we incorporate Explainable AI (XAI) techniques to enhance model interpretability and practical usability.
    \item \textbf{Predictive Relevance of IV and RV}: We conduct a detailed assessment of Implied Volatility (IV) and Realized Variance (RV) as predictors of one-day-ahead Expected Shortfall (ES) for individual equities, offering new insights into their individual and combined predictive contributions.
\end{enumerate}

Together, these contributions advance the methodological and empirical frontiers of financial risk forecasting by demonstrating the effectiveness of probabilistic AI models and offering new insights into model performance, uncertainty, and interpretability in equity markets. The findings also have practical relevance, as they demonstrate the possibility of improving risk management practices.


\begin{comment}
    
This research makes several key contributions to the field of financial forecasting and risk management by leveraging probabilistic AI models for full distributional forecasts and tail risk estimation. The primary contributions are as follows: 

\begin{enumerate}
    \item \textbf{Flexible Distributional Modeling:} We implement and evaluate probabilistic AI models, specifically LSTM-MDN and Transformer-MDN, that move beyond restrictive parametric assumptions. These models conditionally adapt to market dynamics, allowing the data to dictate distributional forms, effectively capturing heavy tails, skewness, and conditional volatility dynamics.
    \item \textbf{Enhanced Tail Risk Estimation:} Our approach produces full, non-parametric conditional return distributions using MDNs, enabling analytical calculation of Expected Shortfall (ES). We demonstrate that our model achieves statistical adequacy and improved performance over all benchmark models on key scoring metrics, offering a more accurate alternative to tail risk estimation.
    \item \textbf{Quantification and Differentiation of Uncertainty:} We systematically quantify and differentiate aleatoric (data-driven) and epistemic (model-driven) uncertainty, providing deeper insights into the source of uncertainty, improving prediction reliability and underlying risk assessment.
    \item \textbf{Empirical Benchmarking and Comparative Analysis:} We establish a comprehensive benchmarking framework, comparing probabilistic AI models against traditional econometric models and prevalent machine learning models. As part of this, we extend and apply the booster models developed by \textcite{moen2024forecasting} (XGBoost, LightGBM, CatBoost) to a new asset class and adapt them for multi-asset prediction, thereby building upon and enhancing prior research. We implement a rigorous evaluation framework, including distributional accuracy measures, statistical adequacy tests, backtesting procedures, and tail risk scoring. This ensures that model performance is assessed reliably and comprehensively.
    \item \textbf{Comparative Evaluation of AI Architectures:} We systematically evaluate and compare the probabilistic forecasting capabilities of LSTM-MDN and Transformer-MDN architectures. Specifically, we address the ongoing debate in the literature regarding the utility of Transformers in financial time series forecasting by explicitly assessing whether Transformers enhance predictive performance and tail risk estimation compared to LSTMs. %highlighting the conditions under which each architecture performs best.
    \item \textbf{Explainability and Interpretability in AI Models:} Recognizing the critical need for transparency in financial risk management, particularly as machine learning models are often criticized for their black-box nature compared to econometric methods, we incorporate Explainable AI (XAI) techniques. This enhances the interpretability and practical applicability of the models.
    \item \textbf{Predictive Power of IV and RV Measures:} We provide a comprehensive evaluation of Implied Volatility (IV) and Realized Variance (RV) as explanatory variables in forecasting one-day-ahead ES for individual equities. This evaluation contributes novel insights into their relative and combined efficacy.
\end{enumerate}

\end{comment}


%%% OLD POINTS %%%%%

\begin{comment}
    
\begin{enumerate}
    \item Flexible Distributional Modeling: We implement and evaluate probabilistic AI models that eliminate restrictive parametric assumptions of traditional econometric models. The proposed models conditionally adapt to changing market conditions and let the data define the distributional form, enabling more accurate modeling of heavy tails and return distributions.
    \item Enhanced Tail risk Estimation: By leveraging models that produce full, non-parametric conditional return distributions, we enhance ES estimation. Full distributions enable analytical ES calculation, offering a more accurate and theoretically sound alternative to the approximation methods used in traditional approaches. 
    \item Quantifying and Differentiating Aleatoric and Epistemic Uncertainty: The research systematically distinguish between aleatoric (data-driven) and epistemic (model-driven) uncertainty in financial forecasting, providing deeper understanding and insights of model reliability and risk assessment.  
    \item Empirical Outperformance of Traditional and ML Models: We conduct a comprehensive empirical comparison of the employed probabilistic AI models against traditional econometric models and widely used machine learning models. The analysis demonstrate superior performance in distributional accuracy, quantile estimation precision and tail risk quantification. 
    \item Comparative Analysis of AI Architectures: We systematically evaluate the performance of different probabilistic AI architectures—LSTM-MDN and Transformer-MDN, and VAE-MDN—in capturing heavy-tailed distributions and improving uncertainty estimation in financial time series forecasting.
    \item Robust Benchmarking Framework: Unlike prior studies employing probabilistic AI models, which often rely on simple accuracy metrics and lack rigorous statistical validation, this thesis implements a comprehensive evaluation framework. Our approach includes distributional accuracy measures and statistical adequacy tests, backtesting  and accuracy scoring of tail risk estimates. This ensures that the performance of the proposed models is assessed reliably.
    \item AI Model Explainability: Recognizing the importance of transparency and regulatory compliance in financial risk management, we explore Explainable AI (XAI) techniques to enhance the interpretability and trustworthiness of probabilistic AI models in tail risk estimation.  
    \item Something something about testing the models developed by Utne et al on a different asset class as well as rewriting them for multi-asset forecasting. 
    \item Something something about testing the efficacy og IV and RV and combuination for tail risk estimation
    \item Something something integrate with the benchmarking point about testing several ML models 
    \item Something something only article to test prob AI models on ES estimate
    \item IV in ES prediction more or less not done

\end{enumerate}

These contributions significantly advance academic research by addressing key gaps in the literature related to the use of probabilistic AI for distributional forecasting and tail risk estimation in financial time series analysis.

\end{comment}

% ==================== Thesis Structure ==================== %
\subsection{Thesis Structure}
\label{sec:thesis_structure}

The remainder of this thesis is structured as follows. Section \ref{sec:2_literature} provides an overview of existing literature on traditional approaches to VaR and ES estimation, uncertainty quantification and tail risk estimation using probabilistic AI in finance, applications of the Transformer architecture in financial forecasting, and the comparative use of Implied Volatility (IV) and Realized Variance (RV) in forecasting financial tail risk. Section \ref{sec:4_data} describes the dataset, detailing sources, feature engineering, and preprocessing procedures. Section \ref{sec:5_methods} presents the methodological approach, including the specification and implementation details of the probabilistic AI models, benchmark models, evaluation metrics, and testing procedures. Section \ref{sec:6_results} reports empirical findings, evaluating distributional accuracy, performance in tail risk estimation, and the relative predictive capabilities of IV and RV. Section \ref{sec:7_discussion} discusses the implications of our findings, highlighting benefits, limitations, and potential directions for future research. Finally, Section \ref{sec:8_conclusion} concludes by summarizing the key insights and the significance of probabilistic AI for financial tail risk estimation and risk management.