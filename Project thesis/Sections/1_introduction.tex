% Fix wrong first number for footnote
\setcounter{footnote}{0}

\section{Introduction}
\label{sec:introduction}
%\textit{Briefly touch upon the need for better uncertainty estimates in finance and how probabilistic AI models have started to fill gaps left by traditional econometric approaches.}

In recent years, there has been a significant increase in the use of artificial intelligence (AI)\footnote{See Appendix \ref{appendix:list_of_abbrevations} for a list of all abbreviations used in this paper.} and machine learning (ML) models within finance, motivated by their potential to provide price forecasts even in efficient markets \parencite{sezer2020financial}. However, a major drawback of AI models is that they are to a large extent ``black boxes'', making it difficult to understand how to interpret predictions or trading signals. Probabilistic AI models partly alleviate this issue by providing well-calibrated probabilities for different scenarios, allowing for more informed decision-making and sophisticated investment strategies. However, as this review will show, the usefulness of these probability estimates varies widely based on implementation.

Despite the increasing interest in AI-driven approaches, financial time series modeling remains largely dominated by traditional econometric approaches \parencite{lopez2019beyond_econ}. Established methods such as autoregressive integrated moving average (ARIMA) models \parencite{boxJenkins2016time} dominate in prediction of future values, i.e. the conditional mean of the time series. However, the Efficient Market Hypothesis (EMH) posits that the conditional mean cannot be predicted, as all information is already incorporated into the price, and econometric models that attempt this thus have very limited predictive power, if any \parencite{Campbell2007}. Nevertheless, the EMH does not preclude the possibility of predicting the conditional variance, essential for risk management and portfolio optimization. Generalized autoregressive conditional heteroskedasticity (GARCH) models \parencite{BOLLERSLEV1986GARCH}, a widely used family of models for volatility prediction (i.e., predicting the conditional variance of a time series), are commonly employed for this purpose. It is also common to combine the two types of models—e.g. ARIMA and GARCH—to construct a model capable of predicting both the conditional mean and the conditional variance. Still, econometric models such as ARIMA and GARCH remain constrained by their simple parametric and linear structure, restricting their ability to capture the non-linear and complex interactions often observed in financial data \parencite{sezer2020financial}. 

In contrast, AI models such as Long Short-Term Memory networks (LSTM) and Convolutional Neural Networks (CNN) have recently emerged as alternatives for handling non-linearity within data for financial prediction. These models have demonstrated potential in capturing these complexities, but have primarily been used to provide point forecasts rather than full conditional probability distributions \parencite{Tang2022Survey}. This limits both the interpretability and the practical usefulness of their predictions. This survey examines whether applying probabilistic AI models' to enhance uncertainty estimates in financial time series prediction can alleviate these issues, and their potential value for financial risk interpretation, decision-making and portfolio risk management among other applications. 


% Context 
\textbf{Context}\nopagebreak

Probabilistic AI often refers to a Bayesian approach to machine learning, where probabilistic theory is used to infer plausible models to explain observed data \parencite{Ghahramani2015}. However, to capture all the research relevant for improving uncertainty estimates in finance, we extend the definition to include any machine learning models that can quantify the uncertainty in their predictions or produce full distributional forecasts.

Probabilistic AI is a rapidly evolving field, as this review will demonstrate, and includes models like Bayesian neural networks, Variational Autoencoders, Gaussian processes, and Probabilistic Neural Networks among others. The number of papers applying probabilistic models in a financial context has risen sharply the last years, with varying reported results. However, the majority of AI applications in finance have focused on generating precise point forecasts for prices or returns, and the potential usefulness of probabilistic predictions has received less attention \parencite{sezer2020financial}.

When discussing quantification of prediction uncertainty, it is important to distinguish between epistemic (model-driven) and aleatoric (underlying) uncertainty. Epistemic uncertainty arises from the model's limitations in capturing data patterns, caused by not having enough data, not using the relevant features or misspecifying the model. This type of uncertainty is reducible through better models or more data. Conversely, aleatoric uncertainty refers to inherent data randomness, equivalent to latent volatility in finance. It is influenced by unpredictable market behavior, economic events, and investor sentiment, beyond the reach of any model \parencite[7,34]{pml1Book, KIUREGHIAN2009105, hullermeier2021aleatoric}. The total uncertainty is thus the sum of the epistemic variance and the aleatoric variance, given by the predictive variance decomposition \parencite{depeweg2018uncertainty_decomposition}:
\begin{equation}
    \begin{footnotesize}
    \begin{gathered} 
    V(y^*|x^*,\mathcal{D})=
    \underbrace{\mathbb{E}_{\mathbf{w}\sim p(\mathbf{w}|\mathcal{D})}\big[V(y^*| x^*,\mathbf{w})\big]}_{\displaystyle\text{aleatoric uncertainty}}+
    \underbrace{V_{\mathbf{w}\sim p(\mathbf{w}|\mathcal{D})}\big[\mathbb{E}(y^*| x^*,\mathbf{w})\big]}_{\displaystyle\text{epistemic uncertainty}}
    \end{gathered}
    \end{footnotesize}
\end{equation}
where $y^*$ is the prediction for a new input $x^*$, $\mathcal{D}$ is the training dataset of observed input-output pairs $\{(x_i,y_i)\}^{N}_{i=1}$ and $\mathbf{w}$ denotes the model weights, treated as a random vector following the posterior distribution $p(\mathbf{w}|\mathcal{D})$.

A sophisticated probabilistic AI model can quantify both types of uncertainty, as well as model how the epistemic and the aleatoric uncertainty—i.e. the volatility—changes over time. However, as this review will show, only a minority of proposed models fully exploit this possibility.

Probabilistic AI models have broad applications in investment and risk management. For example, an investor may seek to identify stocks likely to outperform the market. However, in highly efficient markets, even advanced predictive models can only provide recommendations that are marginally more accurate than random guesses, making it challenging for an investor to understand how to interpret predictions. In such cases, probabilistic AI models are valuable as they estimate predictive uncertainty, allowing investors to make more informed decisions. Moreover, these models' ability to differentiate between aleatoric and epistemic uncertainties further enhance their utility. For instance, an investor might construct a portfolio to maximize returns relative to aleatoric uncertainty, while simultaneously imposing a threshold to exclude predictions where epistemic uncertainty is unacceptably high. Similarly, the investor could use the model’s output to develop specialized risk measures that integrate predicted returns with both types of uncertainties, allowing for a broad range of sophisticated investment strategies. In sum, probabilistic AI models provide a versatile and flexible framework for analyzing financial time series, ultimately facilitating more informed, adaptive, and robust investment decisions.

% Overview of research area

\textbf{Overview of Research Area}\nopagebreak

There have been several studies and reviews over the last years on the application of AI and machine learning for financial time series prediction. \textcite{gandhmalstockmarket2019}, \textcite{Li2020} and \textcite{Kumbure2022} review machine learning techniques applied to stock market trend or point predictions, up to 2018, 2019 and 2019 respectively. \textcite{gandhmalstockmarket2019} conclude that Artificial Neural Networks (ANNs) and fuzzy-based techniques are the most promising among the reviewed machine learning approaches for accurate stock market predictions, supported by \textcite{shi2019soft} review of soft computing approaches, finding ANN architectures to consistently outperform other machine learning models in point prediction accuracy. Li and Bastos (2020) and \textcite{sezer2020financial} show that RNN based models like LSTM implementations are the most popular in deep learning. \textcite{Kumbure2022} conclude that the most frequently utilized models are ANNs and SVMs, but that deep learning models like LSTM have growing interest due to reports of robust and improved predictions. 

A newer study conducted by \textcite{Khattak2023SurveyAIModels} provides an in-depth review of machine learning methods applied to forecast various financial assets between 2018 and 2023. The study finds new hybrid integrations of LSTM and SVM architectures most effective in financial predictions, outperforming traditional models in point accuracy.

While there is a growing body of literature on AI and machine learning for financial applications, the aforementioned reviews focus solely on point prediction accuracy. Even though both epistemic and aleatoric uncertainty are essential components of financial decision making, fewer reviews have been conducted on the topic of uncertainty quantification in a financial context using machine learning methods or probabilistic models. \textcite{abdar2021ReviewUQ} conduct a review on uncertainty quantification using deep learning techniques, discussing the advantages and disadvantages of several models, but do not focus on financial time series predictions. The authors conclude that Deep Ensembles and Bayesian Neural Networks show promising capabilities for uncertainty quantification distinguishing between aleatoric and epistemic uncertainty, applicable to financial forecasting, but find that lack of standardized benchmarks make it difficult to compare frameworks.

\textcite{Blasco_et_al_2024} conduct a survey on uncertainty quantification using deep learning techniques in financial time series. The study shows that most papers do not distinguish between aleatoric and epistemic uncertainty, and few authors perform analysis on the financial implications of predictive uncertainty. The review is limited to deep learning models using a Bayesian approach, and investigates methods for approximation of posterior distributions in these models. Their focus on how to use and interpret uncertainty estimates in a financial context is limited, and they do not assess how researchers evaluate uncertainty estimates or discuss the appropriate way of doing so. Specifically, they do not assess whether probabilistic AI models truly are an improvement over econometric models and traditional AI models when it comes to their practical utility in financial applications and decision-making. The survey includes literature on deep learning up to 2022, but as we show in this review, production has seen a dramatic increase over the last two years. In our survey, we expand the scope to address these gaps, while also using a broader definition of probabilistic AI.

In light of the rapid increase in published papers on probabilistic AI over recent years and the scarcity of reviews focused on uncertainty quantification for various financial time series, this review aims to synthesize the current state-of-the-art research. Additionally, prior financial studies have predominantly concentrated on stock markets; in response, this review includes a range of asset classes, examining how uncertainty quantification varies across categories. Given that the defining feature of probabilistic AI models is their capacity to provide predictions with associated uncertainties, this review places substantial emphasis on interpreting these uncertainty estimates within a financial context. Finally, existing research reveals a lack of consensus regarding the evaluation of uncertainty estimates produced by probabilistic models. Accordingly, we summarize current evaluation practices and, based on this, propose a clear framework for future work. 


% Reserach questions
\textbf{Research Questions}\nopagebreak

To gain an understanding of the existing research, address the gaps identified in the literature, and advance the understanding of the use of probabilistic AI for uncertainty estimates in financial time series, this review seeks to answer the following research questions: 
\begin{enumerate}[label=RQ\arabic*:]
    \item Summarize to what extent and in what way existing research are using uncertainty estimates from probabilistic AI models as measures of volatility, model uncertainty, or financial risk.
    \item Analyze researchers' motivation for making predictions with uncertainty and how it differs for different asset classes.
    \item Compare how promising probabilistic models' capabilities are compared to other machine learning and traditional econometric models.
    \item Investigate probabilistic models' ability to provide improved understanding of risk and volatility compared to econometric models.
    \item Identify the metrics used for assessing probabilistic AI models and assess what the most appropriate metrics for assessing the quality of the produced uncertainty estimates are.
    \item Analyze how probabilistic AI models can be used to construct financial risk measures such as Value at Risk (VaR) and Expected Shortfall (ES).
    \item Identity possible areas for further research.
\end{enumerate}

The questions will be answered directly in Section \ref{sec:discussion}, but they will shape the presented results throughout the review. 

 
% Contributions
% \textit{How review addresses the specific applications in finance and recent advancements in probabilistic AI}
\textbf{Contributions}\nopagebreak

This review addresses a significant gap in the literature by systematically analyzing the application of probabilistic AI models in finance, with a focus on their potential to improve uncertainty estimation in financial forecasting. While probabilistic AI is a rapidly evolving field, its adoption and understanding within financial contexts remain limited. This review contributes to the field by offering a comprehensive overview of recent advances and identifying key strengths and limitations of the models in practice.

This study examines how probabilistic AI models may improve uncertainty estimates by clarifying the role of aleatoric (data-driven) and epistemic (model-driven) uncertainties, and assesses their relevance for financial decision-making. It also evaluates whether these models offer improvements over traditional econometric approaches like GARCH in quantifying and interpreting risk and volatility.

Key contributions include:

\begin{enumerate}
    \item Systematically Examining How Probabilistic AI is Used in Finance: We examine whether quantified uncertainty from probabilistic AI models can be interpreted as volatility estimates, providing insights into their suitability for financial risk measures such as Value at Risk (VaR) and Expected Shortfall (ES).
    \item Critical Evaluation of the State of the Art: We critically assess whether the current research in probabilistic AI represents an improvement over traditional models for financial time series prediction.
    \item Benchmarking and Evaluation: The review identifies best practices and gaps in benchmarking probabilistic AI models against traditional approaches, highlighting the need for standardized evaluation frameworks for predictive accuracy and uncertainty quantification.
    \item Identification of Most Widely Used Models: We categorize the probabilistic machine learning models most widely used in finance, such as Bayesian Neural Networks (BNNs), Gaussian Processes (GPs), and Probabilistic Recurrent Neural Networks (RNNs), and evaluate their potential for further research and application.
    \item Practical Guidance for Researchers: The review emphasizes how adopting probabilistic AI models can advance financial analysis by incorporating both point predictions and full distributional forecasts, enabling improved risk assessment, portfolio optimization, and market simulation.
    \item Roadmap for Future Research: We highlight key areas for advancing uncertainty
quantification in finance, including:
        \begin{itemize}[leftmargin=1.5em] % Adjust '2em' to control indentation
            \item Using non-parametric distributions from probabilistic AI models to improve risk measures
            \item Separating epistemic from aleatoric uncertainty when using AI models with uncertainty quantification
            \item Establishing a standardized testing and benchmarking framework to prove that new proposed models are significant improvements over existing traditional models, while also establishing a practice of making code available so that the field can advance by successively building upon and testing against previously proposed models
            \item More inter-disciplinary collaboration to ensure that proposed models appropriately incorporates financial knowledge and state-of-the-art AI techniques
        \end{itemize}
\end{enumerate}

Ultimately, this review aims to serve as a foundational resource and guide for researchers and practitioners looking to apply or extend probabilistic AI techniques within finance, bridging the gap between machine learning advancements and financial applications.

% Concluding paraghaph outlining the structure of the paper 
\textbf{Structure of the Paper}\nopagebreak

The literature review is structured as follows: Section \ref{sec:methodology} covers the Methodology, detailing the review and analysis process. Section \ref{sec:descriptive_statistics} presents Brief Descriptive Statistics, while Section \ref{sec:result_and_discussion} presents the Results and Discussion, including evaluation across different dimensions. Finally, Section \ref{sec:conclusion} provides the Conclusion, summarizing key findings and suggestions for future research.  