\label{sec:abstract}

%"probabilistic AI to improve uncertainty estimates in financial time series" , " uncertainty quantification", "volatility", "Artificial intelligence" and "machine learning", "aleatoric and epistemic uncertainty", "Financial time series" , "64 papers", "traditional models", "financial risk measures", "To ensure generalizability and validity we will use data from a broad range of markets.", ARIMA, GARCH, VaR (Value-at-risk)

\normalsize

In this systematic literature review, we examine the application of probabilistic AI models to improve uncertainty quantification in financial time series forecasting. Unlike traditional AI models, probabilistic AI approaches are capable of quantifying the uncertainty in predictions and providing full distributional forecasts. Furthermore, sophisticated probabilistic AI frameworks can differentiate between aleatoric and epistemic uncertainty, enhancing financial interpretation and improving associated risk assessments. We systematically review recent advancements across various probabilistic frameworks—including Bayesian neural networks, Gaussian processes, and generative models—applied to a range of financial assets. Findings are categorized by model type, output format, asset class, and the type of uncertainty quantified.

Our analysis reveals significant gaps in the field, including a lack of standardized benchmarks, evaluation metrics, and interdisciplinary collaboration, as well as limited financial interpretation of results. Few studies rigorously validate uncertainty estimates, and even fewer benchmark satisfyingly against traditional models such as GARCH, limiting the ability to draw definitive conclusions about the performance of probabilistic models. Our findings suggest that these shortcomings may stem from the predominance of computer scientists over finance researchers in the field. The field appears to be new and fragmented, as most papers have been published recently and few authors build upon existing research, possibly because code is rarely disclosed.

We conclude that while probabilistic AI has significant potential to enhance risk assessment and financial decision-making through better uncertainty quantification, this potential remains largely underutilized. To address these issues, we propose a standardized evaluation framework and advocate for stronger interdisciplinary collaboration to advance the application of probabilistic AI in financial forecasting.

%Ultimately, this review aims to serve as a foundational resource and guide for researchers and practitioners looking to apply or extend probabilistic AI techniques within finance, bridging the gap between machine learning advancements and financial applications.


